\chapter*{Task Allocation}

\subsection*{Definition}

Task Allocation is the problem of deciding "who does what, where, when and how" in a multi-robot system.

It involves finding a mapping / allocation A: T → R, which assigns a set of $n_t$ tasks (T) to a set of $n_r$ robots (R).

The goal is to find an assignment that maximizes overall global system utility.

\subsection*{MRTA Taxonomy}

\begin{table}[h!]
\centering
\renewcommand{\arraystretch}{1.2}
\begin{tabularx}{\textwidth}{@{}>{\centering\arraybackslash}p{3cm}|
                                  >{\centering\arraybackslash}p{2.5cm}|
                                  >{\centering\arraybackslash}p{2.5cm}|
                                  X@{}}
\toprule
\textbf{Dimension} & \textbf{Type 1} & \textbf{Type 2} & \textbf{Description} \\ \midrule
Robot Tasking (R) & Single-Task (ST) & Multi-Task (MT) & Can a single robot work on one task (ST) or multiple tasks simultaneously (MT)? \\ \addlinespace
Task Type (T) & Single-Robot (SR) & Multi-Robot (MR) & Does a task require one robot (SR) or a coalition/team of robots (MR)? \\ \addlinespace
Assignment Time (A) & Instantaneous Assignment (IA) & Time-Extended Allocation (TA) & Is the assignment instantaneous or one-shot (IA), or does it involve planning a schedule or route over time (TA)? \\
\bottomrule
\end{tabularx}
\caption{Dimensions of multi-robot task allocation (MRTA) taxonomy.}
\label{tab:mrta-dimensions}
\end{table}

\subsection*{The "Hello World": ST-SR-IA}
The simplest MRTA problem is the ST-SR-IA problem, where each robot can only do one task at a time (ST), each task requires only one robot (SR), and the assignment is instantaneous (IA).

Linear Assignment Problem (LAP) Solution: The optimal assignment can be found in polynomial time using the Hungarian Algorithm with a complexity of O($n^3$), (where n is the number of robots/tasks).

\begin{table}[h!]
\centering
\renewcommand{\arraystretch}{1.2}
\begin{tabularx}{\textwidth}{@{}p{2.5cm}|p{6.7cm}|X@{}}
\toprule
 & \textbf{Quick Definition} & \textbf{Important Result / Insight} \\ \midrule
\textbf{Adding Task Priorities} & Tasks have priority weights \(w_t\) indicating importance, added to the objective function. & Weighted assignment maximizes total utility \(\sum_r \sum_t w_t U_{rt} x_{rt}\). \\ \addlinespace
\textbf{$|R| \neq |T|$ (Unequal sets)} & Number of robots and tasks differ. Use dummy robots/tasks to balance. & Two IA rounds preserve polynomial-time solvability without affecting optimal real assignments. \\ \addlinespace
\textbf{Iterated Assignment} & Utilities or task info change over time → recompute or adaptively update assignments. & BLE (2001): 2-competitive ($\geq$50\% of optimal); L-ALLIANCE (1998): learns best assignments iteratively. \\ \addlinespace
\textbf{Online Assignment} & Tasks revealed one-by-one; robots may not be reassignable. & MURDOCH (2002): greedy, 3-competitive; best possible for online assignment without reassignment. \\
\bottomrule
\end{tabularx}
% \caption{Summary of key ST–SR–IA concepts in multi-robot task allocation.}
\label{tab:st-sr-ia-summary}
\end{table}
