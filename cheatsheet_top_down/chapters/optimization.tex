\section*{Optimization}

Comparing PSO with single optimizing agents. We saw for PSO 
if left alone, each agent would behave like a memory-based random searcher,
biased toward its own best-so-far. 

A conservative blind explorer, with no reliable signal to use for improving.

\section*{Hill-Climber Optimizing Agent}

Smarter Optimizing Agent. Like climbing Everest in thick fog with amnesia.

\begin{tightitemize}
    \item Start at a random point
    \item Sample the neighborhood of the current point
    \item Always moves in direction of local imporovement
    \item Repeat until no neighbor is better (it has reached a peak).
\end{tightitemize}

\textbf{Variants to improve it:}
\begin{tightitemize}
    \item Stochastic hill-climber: introduces randomness in the selection of neighbor points
    \item Random Restarts: escape mechanism for local optima
\end{tightitemize}

Hill-climbing is powerful for Constraint Satisfaction Problems (CSPs), like n-Queens.
It can solve large instances of $n$-queens ($n$ = 106) in a few seconds

\section*{Gradient-Based Agents}

\textbf{The Gradient ($\nabla f(\mathbf{x})$)}\\
The gradient is the vector of partial derivatives 
$
\nabla f(\mathbf{x}) = \left(\frac{\partial f}{\partial x_1}, \dots, \frac{\partial f}{\partial x_n}\right),
$ 
and it points in the direction of the steepest increase of the function $f$ at $\mathbf{x}$.

\vspace{0.5em}
\textbf{Gradient Descent Algorithm}\\
To minimize a function, an agent iteratively moves in the direction of the negative gradient:

\begin{tightitemize}
    \item Start at an initial point $\mathbf{x}_0$.
    \item Compute the descent direction: $\mathbf{d}_k = -\nabla f(\mathbf{x}_k)$.
    \item Choose a step size $\alpha_k$ (how far to move).
    \item Update the position: $\mathbf{x}_{k+1} = \mathbf{x}_k + \alpha_k \mathbf{d}_k$.
    \item Repeat until the gradient is approximately zero ($\nabla f(\mathbf{x}) \approx 0$), indicating a minimum.
\end{tightitemize}

For convex functions we are guaranteed to get to the local and global minimum 
in a finite (possibly small) number of iterations

For non-convex functions, the final local optimum depends on where we start from

\section*{Applications for ML}

Optimization function represents the classification Error of a ML model during training,
i.e., the total Loss on on training set 
$
\sum_{i=1}^{m}l(\hat{y}^i(\ww^i, \xx^i), y^i)
$

\textbf{Gradient Descent Procedure for OLSR}

For a linear model with quadratic loss:
$
E(\mathbf{w}; X, Y) = \frac{1}{2} \sum_{i=1}^N (\mathbf{w}^\top x_i - y_i)^2
$

The gradient is:
$
\nabla E(\mathbf{w}) = \sum_{i=1}^N (\mathbf{w}^\top x_i - y_i)x_i
$

Gradient descent update:
$
\mathbf{w}^{(t+1)} = \mathbf{w}^{(t)} - \alpha \sum_{i=1}^N (\mathbf{w}^{(t)\top} x_i - y_i)x_i
$

\begin{tightitemize}
    \item Initialize $\mathbf{w}^{(0)}$ (e.g., random or zero vector), $t \gets 0$
    \item Repeat until convergence or maximum iterations $N$:
          Compute gradient $\nabla E$, choose step size $\alpha$, update parameters:  
          $\mathbf{w}^{(t+1)} = \mathbf{w}^{(t)} - \alpha \nabla E$,  
          break if $\|\mathbf{w}^{(t+1)} - \mathbf{w}^{(t)}\| <$ tolerance
    \item Return $\mathbf{w}^{(t+1)}$ as the solution
\end{tightitemize}


\section*{Batch vs Stochastic GD}

\begin{tightitemize}
    \item Batch Mode (Regular): Gradient at each step is computed over the entire function,
    This is accurate but impossibly slow if $m = 1$ billion.
    \item Stochastic Gradient Descent: Gradient at each step is
    computed over one single component / data point selected out of the $m$ ones.
    Takes many more steps (but each step is computationally light). The steps are very fast but "erratic".
    \item Mini-batch (stochastic) Gradient Descent: Gradient at each step is computed
    over one subset of data points. This is good balance of speed and stability.
\end{tightitemize}

\section*{Momentum and Adam}

SGD can be slow or get stuck

\textbf{Momentum (MGD)}\\
Adds an inertia term by combining the current gradient with the previous update, allowing the agent to roll through small bumps and accelerate along gentle slopes:
$
\mathbf{v}_t = \gamma \mathbf{v}_{t-1} + \alpha \nabla f(\mathbf{w}_t), \quad
\mathbf{w}_{t+1} = \mathbf{w}_t - \mathbf{v}_t
$

\textbf{Adam (Adaptive Moment Estimation)}\\
A popular optimizer combining two ideas:
\begin{tightitemize}
    \item \textbf{Momentum ($\mathbf{m}_t$)}: exponentially weighted average of past gradients (1st moment) to determine direction.
    \item \textbf{Adaptive scaling ($\mathbf{v}_t$)}: exponentially weighted average of past squared gradients (2nd moment) to adapt learning rates per parameter.
\end{tightitemize}

\section*{Comparing ADAM against PSO}

One core difference: ADAM is a Local Optimizer, PSO is a Global optimizer!

In general Adam worse, unless the shape is nice (Adam did well on ellispoid, got smoked in 
griewank, rastrigin, schwefel)

\section*{Hybrid?}

If PSO is a great "explorer" and Adam is a great "exploiter," the obvious solution is to combine them.
    
The Strategy:
\begin{tightitemize}
    \item PSO Phase: Let the PSO swarm run for several iterations. 
    The particles spread out and find the most promising valleys.
    \item Local Search Phase: 
    Take the best particles and run a local optimizer starting from their positions.
    \item Cooperation: The local optimizer  will find the exact bottom of that valley. 
    This new accurate position is then fed back to the swarm as the new gbest.
\end{tightitemize}
        

        

        